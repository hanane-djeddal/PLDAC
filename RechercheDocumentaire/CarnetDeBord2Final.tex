% pour genree un pdf: faire
% pdflatex exemple.tex
\documentclass{article}

%% Paquets LateX utiles

\usepackage[utf8]{inputenc} 		% encodage des caracteres utilise (pour les caracteres accentues) -- non utilise ici.
%\usepackage[latin1]{inputenc} 		% autre encodage
\usepackage[french]{babel}		% pour une mise en forme "francaise"
\usepackage[T1]{fontenc} 
\usepackage{amsmath,amssymb,amsthm}	% pour les maths
\usepackage{graphicx}			% pour inclure des graphiques

\usepackage[hidelinks]{hyperref}
\usepackage{color}			% pour ajouter des couleurs dans vos textes
\usepackage{geometry}
\geometry{hmargin=2.5cm,vmargin=3cm}
\renewcommand{\contentsname}{\centering Contents}


\begin{document}
\begin{titlepage}
    \begin{flushleft}
    \includegraphics[width=11em]{logo.png}\\[1.5cm]
    \end{flushleft}
    \begin{center}
        \textsc{{\LARGE \color{blue} Master Données, Apprentissage et Connaissances-DAC}}\\[5cm]
        \textsc{\Huge{CARNET DE BORD}}\\[1cm]
        \textbf{\Large{Sujet:}}
        \textsc{\large{Problème de clustering pour infrastructures sans fil.}}\\[6cm]
        \begin{minipage}{1\textwidth}
            \begin{flushleft} \large
            \textsc{\LARGE{Realisé par :}}\\[0.5cm]
            \textsc{Liticia TOUZARI}\\
            \textsc{Hanane DJEDDAL}\\[0.5cm]
            \textsc{Spécialité : DAC }\\ [1.5 cm]
            \end{flushleft}
        \end{minipage}
        \vfill
    \end{center}
  \end{titlepage}
  

\tableofcontents%  Table des matieres


\newpage

\vspace*{\stretch{0.5}}
  \begin{center}
\section*{\LARGE{Introduction}}
  \end{center}
\Large{\paragraph{}
        Aujourd'hui, le trafic de données sur les réseaux mobiles connaît une croissance explosive à mesure que les smartphones et tablettes compatibles avec Internet deviennent de plus en plus populaires.\\
Afin de répondre à la demande croissante de trafic de données, les opérateurs de réseaux mobiles doivent augmenter leur capacité de traitement de données, comme le déploiement de plus de stations de base et l'ajout d'unités de traitement de données supplémentaires aux stations de base. \\
Cependant, les dépenses en capital liées au déploiement de ces infrastructures de réseau deviennent de plus en plus élevées. Par conséquent, l'optimisation des dépenses d'investissement et des dépenses d'exploitation tout en maintenant une qualité de service est devenue une nécessité pour les opérateurs de réseaux mobiles.
\paragraph{}
Ce projet vise à mettre en œuvre des techniques de clustering afin de regrouper les ressources et infrastructures radio dans le but d'améliorer l'efficacité des services. \\
Dans ce cadre, on proposera des méthodes de clustering efficaces, adaptées aux spécificités de l'environnement sans fil et des métriques des services de télécommunication. On utilisera des données fournies par l'opérateur Orange.
}
\vspace*{\stretch{1}}
\newpage

\section{Travail attendu}
\paragraph{}
 Le projet est divisé en une séquence de trois taches principales à réaliser : \\
\begin{itemize}
    \item Étude bibliographique et état de l'art. 
    \item Proposition de différents schémas de clustering pour améliorer la qualité de service et l'équité des ressources. En incluant des aspects de performance télécom aux méthodes préexistantes.
    \item Application des méthodes proposées de clustering aux données réelles de telecom afin d'évaluer leur efficacité dans les exemples de réseaux réels.
\end{itemize}

\section{Mots clés retenus}
\begin{flushleft}
Le schéma ci-dessous représente les mots clés utilisés dans ce projet sous forme d'une carte heuristique :\\
\includegraphics[width=38em]{motcle.png}\\[1.5cm]
\end{flushleft}
\section{Descriptif de la recherche documentaire}
\paragraph{}
La recherche documentaire a était principalement divisée en deux grandes parties : 
l'étude de l'architecture C-RAN et l'étude des algorithmes de clustering.\\
Les articles fournis par notre encadrant étaient le point de départ et la référence 
de base qui a guidé la recherche. \\
Dans un premier temps, le but était d’avoir une compréhension général du domaine : Réseau. 
Pour cela, on a utilisé des pages web telles que Wikipedia, Whatis.techtarget, Comment ça marche 
etc, ainsi que des vidéos YouTube pour avoir des définitions des termes, des exemples et des 
explications simplifiées. Par la suite, on a adopté une recherche par mots clés sur google 
scholar, ieeexplore.ieee.org et le moteur de recherche SUper, pour étudier les architectures 
RAN existants et la migration vers le cloud-RAN. On a pu obtenir des ressources telles 
que des thèses de recherches, des revues et des articles de journal qui étaient disponibles en ligne sur les bases de 
données : IEE explore, ScienceDirect et les sites officiels des e-magazines (tel que Heavy Reading de Huawei).
Pour la deuxième partie, l’étude sur les algorithmes de clustering général était basée sur le chapitre 10.3 du livre 
“An introduction to Statistical Learning” et le cours “NDA: Clustering” fourni par notre encadrant; ainsi que les cours de 
l'UE: sciences des données de licence 3 informatique. Ensuite, en suivant la même approche que la première partie, on a 
étudié les algorithmes de clustering proposés pour les architectures C-RAN.
\section{Bibliographie produite dans le cadre du projet}
\paragraph{}Bibliographie générée depuis Zotero avec la norme ACM : \\
\textrm{
\begin{enumerate}
\item[{[1]}] Gabriel Brown. \textit{Cloud RAN & the Next-Generation Mobile Network Architecture}. (2017), 9.
\item[{[2]}] Aleksandra Checko, Henrik L. Christiansen, Ying Yan, Lara Scolari, Georgios Kardaras, Michael S. Berger, and Lars Dittmann. \textit{Cloud RAN for Mobile Networks-A Technology Overview}. IEEE Commun. Surv. Tutorials 17, 1 (2015), 405–426. DOI: https://doi.org/10.1109/COMST.2014.2355255
\item[{[3]}] Longbiao Chen, Dingqi Yang, Daqing Zhang, Cheng Wang, Jonathan Li, and Thi-Mai-Trang Nguyen. 2018. \textit{Deep mobile traffic forecast and complementary base station clustering for C-RAN optimization}. Journal of Network and Computer Applications 121, (November 2018), 59–69. DOI: https://doi.org/10.1016/j.jnca.2018.07.015
\item[{[4]}] Chih-Lin I, Jinri Huang, Ran Duan, Chunfeng Cui, Jesse Jiang, and Lei Li. 2014. \textit{Recent Progress on C-RAN Centralization and Cloudification}. IEEE Access 2, (2014), 1030–1039.\\ DOI: https://doi.org/10.1109/ACCESS.2014.2351411
\item[{[5]}] Gareth James, Daniela Witten, Trevor Hastie, and Robert Tibshirani. 2013. \textit{An Introduction to Statistical Learning}. Springer New York, New York, NY. DOI: https://doi.org/10.1007/978-1-4614-7138-7
\item[{[6]}] Aymen Jaziri, Ridha Nasri, and Tijani Chahed. 2017. \textit{Tracking Traffic Peaks in Mobile Networks Using Statistics of Performance Metrics}. Int J Wireless Inf Networks 24, 4 (December 2017), 389–403. \\DOI: https://doi.org/10.1007/s10776-017-0335-6
\item[{[7]}] Muhammad Khan, Raad S. Alhumaima, and Hamed S. Al-Raweshidy. 2017. {QoS-Aware Dynamic RRH Allocation in a Self-Optimized Cloud Radio Access Network With RRH Proximity Constraint}. IEEE Trans. Netw. Serv. Manage. 14, 3 (September 2017), 730–744. \\DOI: https://doi.org/10.1109/TNSM.2017.2719399
\item[{[8]}] Daewon Lee, Hanbyul Seo, Bruno Clerckx, Eric Hardouin, David Mazzarese, Satoshi Nagata, and Krishna Sayana. 2012. \textit{Coordinated multipoint transmission and reception in LTE-advanced: deployment scenarios and operational challenges}. IEEE Commun. Mag. 50, 2 (February 2012), 148–155. DOI: https://doi.org/10.1109/MCOM.2012.6146494
\item[{[9]}] Shai Shalev-Shwartz and Shai Ben-David. 2014. \textit{Understanding Machine Learning: From Theory to Algorithms}. Cambridge University Press, Cambridge. DOI: https://doi.org/10.1017/CBO9781107298019
\item[{[10]}] John Stillwell. 1997.\textit{Numbers and Geometry}. Springer Science & Business Media.
\item[{[11]}] Jun Wu, Zhifeng Zhang, Yu Hong, and Yonggang Wen. 2015. \textit{Cloud radio access network (C-RAN): a primer}. IEEE Network 29, 1 (January 2015), 35–41. DOI: https://doi.org/10.1109/MNET.2015.7018201
\item[{[12]}] Ziheng Wu and Zixiang Wu. 2020. \textit{An Enhanced Regularized k-Means Type Clustering Algorithm With Adaptive Weights}. IEEE Access 8, (2020), 31171–31179. DOI: https://doi.org/10.1109/ACCESS.2020.2972333
\end{enumerate}
}
\section{Évaluation des sources}
\begin{itemize}
  \item [{[4]}] Recent Progress on C-RAN Centralization and Cloudification :
  \paragraph{}

  Cet article, fourni par notre encadrant, date de 2014, 4 ans après l'introduction 
  de l'architecture C-RAN par China Mobile Research Institute (CMRI). Il donne les 
  derniers progrès dans cette technologie ainsi que les choix d’implémentations et 
  les améliorations possibles. Parmi les auteurs de cet article : la chargée de la 
  recherche et développement des réseaux sans fils avancées de CMRI , ainsi que 2 chefs 
  de projets à CMRI et le directeur de The Green Communications Research Center de CMRI. 
  Le projet étant basé sur le clustering dans les architectures C-RAN, une bonne compréhension 
  de la technologie était un point crucial, et donc cet article, le travail direct d’une équipe 
  de l'Institut où la technologie a émergé, était une source importante pour guider la recherche.
  C'est aussi une source fiable, en effet, l'article est cité 201 sur google scholar, 136 sur web of science, 171 fois sur 
  Scopus et 140 fois sur Crossref. Il était partagé sur des réseaux sociaux tels que Facebook, Twitter et Patents, en plus de  
  103 partage sur Mendeley (un logiciel de gestion bibliographique). Il était utilisé en téléchargement PDF 11131 fois, 
  88 fois étant en janvier 2020. \\
  \item [{[8]}] Coordinated multipoint transmission and reception in LTE-advanced: deployment scenarios and operational challenges
  \paragraph{}
  En un niveau plus détaillé dans la recherche, il était essentiel de comprendre la communications 
  entre RRHs et le fonctionnement des antennes afin de pouvoir critiquer les algorithmes proposés. Cet article 
  explique la technique de coopération Coordinated Multipoint transmission (CoMP) et prévoit l'utilité 
  de cette technique dans les technologies avancées. Publié dans IEEE communications magazine en 2012, 
  cet article est cité 730 fois sur google scholar, 407 fois sur web of science, 474 fois sur Scopus et 447 fois 
  sur Crossref avec un total d'utilisation, qui inclus le téléchargement de PDF et HTML views, de 11108 depuis 
  2012, 54 fois étant en 2020 et 39 fois en 2019 ce qui atteste qu'il est encore d'actualité. L'article,
   qui présente une étude expérimentale était le travail d'une équipe des chercheurs qui ont contribué au
   développement de la technologie LTE et qui viennent de différents groupes spécialisés en télécommunication tels que 
   Orange, Samsung Electronics, LG Electronics, Huawei technologies etc. Ce qui implique une très bonne fiabilité.\\
   \item [{[12]}] An Enhanced Regularized k-Means Type Clustering Algorithm With Adaptive Weights:
   \paragraph{}
  Cet article, disponible sur IEEE digital Library et publié dans le journal IEEE Access en février 2020
  a un lien direct avec l'objectif du projet. En effet, il propose une variante de l'algorithme K-means qui introduit
   la notion de poids, chose qui peut être utile dans le contexte de télécommunication. Selon les métriques de 
   pertinence, l'article n'a pas beaucoup été cité, ce qui est justifié par sa date de publication très récente. Les auteurs 
   sont des nouveaux diplômés de PH.D (2018 et 2019) dans le domaine de l'Electronique. Cependant, le contenu 
   de l'article est, consistant, offrant des démonstrations mathématiques ainsi qu'une évaluation 
   expérimentale. Ses références étaient aussi vérifiables (accessibles en ligne et/ou sur google scholar).

   \vspace*{\stretch{0.5}}
   \begin{center}

\end{document}