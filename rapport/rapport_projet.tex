% pour genree un pdf: faire
% pdflatex exemple.tex
\documentclass{article}

%% Paquets LateX utiles

\usepackage[utf8]{inputenc} 		% encodage des caracteres utilise (pour les caracteres accentues) -- non utilise ici.
%\usepackage[latin1]{inputenc} 		% autre encodage
\usepackage[french]{babel}		% pour une mise en forme "francaise"
\usepackage{amsmath,amssymb,amsthm}	% pour les maths
\usepackage{graphicx}			% pour inclure des graphiques

\usepackage[hidelinks]{hyperref}
\usepackage{color}			% pour ajouter des couleurs dans vos textes
\usepackage{geometry}
\geometry{hmargin=2.5cm,vmargin=3cm}
\renewcommand{\contentsname}{\centering Contents}


\begin{document}
\begin{titlepage}
    \begin{flushleft}
    \includegraphics[width=11em]{logo.png}\\[1.5cm]
    \end{flushleft}
    %\begin{sffamily}
    \begin{center}
        \textsc{{\LARGE \ Master Données, Apprentissage et Connaissances-DAC}}\\[4cm]
        \textsc{\Huge{RAPPORT PROJET DAC}}\\[1cm]
        \textsc{\Huge{Un problème de clustering pour les infrastructures sans fil}}\\[5.5cm]
        %\hspace{30pt}
            % Author and supervisor
        \begin{minipage}{1\textwidth}
            \begin{flushleft} \large
            \textsc{\LARGE{Auteurs :}}\\[0.5cm]
            \textsc{Djeddal Hanane}\\
            \textsc{Touzari Liticia}\\
            \end{flushleft}
        \end{minipage}
        \vfill
    \end{center}
    %\end{sffamily}
  \end{titlepage}
  
\maketitle
\tableofcontents					% si on veut une table des matieres


\newpage


\section{Introduction}
\paragraph{}
    \begin{Large}\\[3.5cm]
        \begin{center}
        %\includegraphics[width=11em]{de.jpg}\\[30cm]
        \end{center}
    \end{Large}
\newpage



\section{Conclusion} 

\end{document}


