%Compilation : pdflatex
\documentclass{report}
\makeatletter
\def\@makechapterhead#1{
\vspace*{50\p@}
  {\parindent \z@ \raggedright \normalfont
    \ifnum \c@secnumdepth >\m@ne
      \if@mainmatter
        %\huge\bfseries \@chapapp\space \thechapter
        \Huge\bfseries \thechapter.\space%
        %\par\nobreak
        %\vskip 20\p@
      \fi
    \fi
    \interlinepenalty\@M
    \Huge \bfseries #1\par\nobreak
    \vskip 40\p@
  }}
\makeatother

\usepackage[utf8]{inputenc}
%\usepackage[T1]{fontenc}
\usepackage[a4paper,left=2cm,right=2cm,top=2cm,bottom=2cm]{geometry}
%\usepackage[frenchb]{babel}
%\usepackage{libertine}
\usepackage[pdftex]{graphicx}
\usepackage{color}

%\setlength{\parindent}{0cm}
%\setlength{\parskip}{1ex plus 0.5ex minus 0.2ex}
\newcommand{\hsp}{\hspace{50pt}}
\newcommand{\HRule}{\rule{\linewidth}{0.5mm}}


%title{Clustering pour les infrastructures sans fils}
%author{Hanane DJEDDAL, Liticia TOUZARI}
%date{2020-02-10}

%\pagenumbering{gobble} Supprimer numéro de page


\begin{document}
\begin{titlepage}
    \begin{flushleft}
    \includegraphics[width=11em]{logo.png}\\[1.5cm]
    \end{flushleft}
    \begin{center}
        \textsc{{\LARGE \color{blue} Master Données, Apprentissage et Connaissances-DAC}}\\[5cm]
        \textsc{\huge{RAPPORT PROJET DAC}}\\[1cm]
        \textsc{\vspace{10pt}\Huge{Clustering pour les infrastructures sans fils}}\\[3cm]
        \begin{minipage}{1\textwidth}
            \begin{flushleft} \large
            \textsc{\LARGE{Realisé par :}}\\[0.5cm]
            \textsc{Hanane Djeddal}\\
            \textsc{Liticia Touzari}\\[1.5 cm]
            \textsc{\LARGE{Encadré par :}}\\[0.5cm]
            \textsc{Anastasios Giovanidis}\\
            \end{flushleft}
        \end{minipage}
        \vfill
    \end{center}
  \end{titlepage}

 
  \vspace*{\stretch{0.5}}
  \begin{center}
    \section*{\LARGE{Résumé}}
\end{center}
    \paragraph{}
  \Large{
    L'augmentation croissante du trafic de données a posé de grands défis aux opérateurs mobiles pour augmenter leur capacité de traitement des données, 
    ce qui entraîne une consommation d'énergie et des coûts de déploiement importants. Avec l'émergence de l'architecture Cloud Radio Access Network (C-RAN), 
    les unités de traitement des données peuvent désormais être centralisées dans les centres de données et partagées entre les stations de base. En mappant 
    un cluster de stations de base avec des schémas de trafic complémentaires à une unité de traitement de données, l'unité de traitement peut être pleinement 
    utilisée à différentes périodes de temps, et la capacité requise à déployer devrait être inférieure à la somme des capacités d'une seule base stations. 
    Cependant, étant donné que les schémas de trafic des stations de base sont très dynamiques à différents moments et endroits, il est difficile de prévoir 
    et de caractériser les schémas de trafic à l'avance pour réaliser des schémas de regroupement optimaux. Dans ce rapport, nous abordons ces problèmes en 
    étudiant les solutions déjà proposées dans le cadre d'optimisation C-RAN basé sur l'apprentissage en profondeur. Premièrement, nous implémentons les algorithmes
    déjà existants, nous procédons par la suite à évaluer leur performances expérimentalement. Nous exposons aussi différents schéma de clustering et nous essayons à les adapter à notre problème. 
    Nous terminons par appliquer les nouvelles méthodes sur des données réelles afin de pouvoir comparer les performances des différents méthodes. \\[2cm]
  \textbf{Mots clés: }
  C-RAN, RAN Cloudification, Clustering
  }
  \vspace*{\stretch{1}}
  

  \tableofcontents
   


  \newpage
  \chapter{État de l'art}
  


  

  \end{document}