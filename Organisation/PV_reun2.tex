% pour genree un pdf: faire
% pdflatex exemple.tex
\documentclass{article}

%% Paquets LateX utiles

\usepackage[utf8]{inputenc} 		% encodage des caracteres utilise (pour les caracteres accentues) -- non utilise ici.
%\usepackage[latin1]{inputenc} 		% autre encodage
\usepackage[french]{babel}		% pour une mise en forme "francaise"
\usepackage{amsmath,amssymb,amsthm}	% pour les maths
\usepackage{graphicx}			% pour inclure des graphiques

\usepackage[hidelinks]{hyperref}
\usepackage{color}			% pour ajouter des couleurs dans vos textes
\usepackage{geometry}
\geometry{hmargin=2.5cm,vmargin=3cm}
\renewcommand{\contentsname}{\centering Contents}


\begin{document}
    \begin{flushleft}
    \includegraphics[width=11em]{logo.png}\\
    \begin{center}
        \textsc{{\color{blue} Master Données, Apprentissage et Connaissances-DAC}}\\[1.5cm]
    \end{center}
    \end{flushleft}
    \begin{center}
        \textsc{\Huge{PV DE RÉUNION}}\\[0.75cm]
    \end{center}
    \textbf{\Large\underline{Projet:}}
    \textsc{\large{Problème de clustering pour les infrastructures sans fil.}}\\[0.75cm]
    \textsc{\Large\underline{Date:}} \large{ Vendredi 21 février 2020.}} \\
    \textsc{\Large\underline{Début de la séance:}} \large{ 10:30.}} \\
    \textsc{\Large\underline{Levée de la séance:}} \large{ 13:30.}} \\
    \textsc{\Large\underline{Étaient présents:}} \large{\\ DJEDDAL Hanane. \\ TOUZARI Liticia.}} \\
    \textsc{\Large\underline{Rapporteur:}} \large{ Djeddal Hanane.}} \\[0.75cm]
    \textbf{\Large\underline{Ordre du jour:}}\\
    \textsc{\large{- Présentation du contenu des articles :ref0a; ref0b et ref1.\\- Compréhension de fonctionnement des antennes réseau. \\- Introdcution de la technologie C-RAN. \\- Discussion des problèmes de la surcharge et inteférences.\\- Introduction de la méthode de clustering des RRHs proposée dans l'article ref1.\\- Discussion sur le critère d'évaluation des clusters.}} \\[0.75cm]
    \textbf{\Large\underline{À préparer pour la prochaine séance:}}\\
    \textsc{\large{- Retour sur l'algorithme de clustering et la formule d'évaluation des clusters proposés dans l'artcile ref1:Comprendre l'algorithme; Critiquer la methode et proposer des améliorations.\\- Lire les références et comprendre l'algorithme K-means.\\-Commencer le rapport du projet.\\- Effectuer une simulation de clustering.\\- Supplémentaires : Lire sur la virtualisation et cloudification, Time series et mobile edge computing.}}\\[0.75cm]
    \textbf{\Large\underline{Date de la prochaine réunion:}}
    \textsc{\large{Vendredi 06 mars 2020.}}



\end{document}


