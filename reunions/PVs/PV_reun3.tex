% pour genree un pdf: faire
% pdflatex exemple.tex
\documentclass{article}

%% Paquets LateX utiles

\usepackage[utf8]{inputenc} 		% encodage des caracteres utilise (pour les caracteres accentues) -- non utilise ici.
%\usepackage[latin1]{inputenc} 		% autre encodage
\usepackage[french]{babel}		% pour une mise en forme "francaise"
\usepackage{amsmath,amssymb,amsthm}	% pour les maths
\usepackage{graphicx}			% pour inclure des graphiques

\usepackage[hidelinks]{hyperref}
\usepackage{color}			% pour ajouter des couleurs dans vos textes
\usepackage{geometry}
\geometry{hmargin=2.5cm,vmargin=3cm}
\renewcommand{\contentsname}{\centering Contents}


\begin{document}
    \begin{flushleft}
    \includegraphics[width=11em]{logo.png}\\
    \begin{center}
        \textsc{{\color{blue} Master Données, Apprentissage et Connaissances-DAC}}\\[0.75cm]
    \end{center}
    \end{flushleft}
    \begin{center}
        \textsc{\Huge{PV DE RÉUNION}}\\[0.75cm]
    \end{center}
    \textbf{\Large\underline{Projet:}}
    \textsc{\large{Problème de clustering pour les infrastructures sans fil.}}\\[0.75cm]
    \textsc{\Large\underline{Date:}} \large{ Vendredi 20 mars 2020.}} \\
    \textsc{\Large\underline{Début de la séance:}} \large{ 10:30.}} \\
    \textsc{\Large\underline{Levée de la séance:}} \large{ 11:50.}} \\
    \textsc{\Large\underline{Étaient présents:}} \large{\\ DJEDDAL Hanane. \\ TOUZARI Liticia.}} \\
    \textsc{\Large\underline{Rapporteur:}} \large{ Djeddal Hanane.}} \\[0.5cm]
    \textbf{\Large\underline{Ordre du jour:}}\\
    \textsc{\large{-Présentation et discussion des résultats de l'application de Clustering K-means sur les données géographiques et l'application de DCCA et DCCA amélioré sur les données du trafic.\\
    - Présentation de dataset du trafic. \\
    - Discussion sur les contraintes physiques liés à la qualité de service: distance max entre RRHs.\\
    - Discussion des améliorations possibles pour le K-means. }} \\[0.25cm]
    \textbf{\Large\underline{À préparer pour la prochaine séance:}}\\
    \textsc{\large{- Analyser les données géographiques Orange : nombre de noeuds par radius, Voronoi diagram.\\
    - Analyser les données du trafic: rédaction des différents metriques. \\
    - Evaluer la valeur de K optimal pour le k-means (en fonction de l'index de DUNN)\\
    - Evaluer la distance acceptable entre deux noeuds afin de respecter les contraintes physiques de qualité de serive. \\
    - Evaluer la complementarité des clusters dans le clustering géographique. \\
    - Appliquer le Clustering Hiérarchique sur les données géographiques.\\
    - Présentation du papier : User-Centric C-RAN Architecture for Ultra-Dense 5G Networks.\\
    - Determiner la capacité  des BBU par rapport au trafic aggrregé normalisé. }}\\[0.25cm]
    \textbf{\Large\underline{Date de la prochaine réunion:}}
    \textsc{\large{Vendredi 03 avril 2020.}}



\end{document}


